%=================================================================-*-LaTeX-*-==
% GRTensorIII 1.50 Manual
% Booklet D: Defining new tensors
%
% Denis Pollney
% July 1996
%==============================================================================
\documentclass{article}
\usepackage{amsmath}
\usepackage{amssymb}
%\usepackage{maple2e}
\usepackage{longtable}
\usepackage{grtensor}
\usepackage{grbooks}
%------------------------------------------------------------------------------
\begin{document}
\grlabel{\grDefLabel}
\grtitle{\grDefTitle}
\grdate{Nov 2016}
\grtitlepage
%==============================================================================
% Body.
%==============================================================================
\copyrightpage
\noindent The GRTensorIII standard library contains the definitions of
a number of tensors, scalars, and other indexed objects whose
components can be calculated. The \grcmd{grdef} command is included to
facilitate the specification of new tensors in a simple and natural
manner.  It allows tensors to be defined either as an equation in
terms of previously defined tensors, or by manual entry of the
component values. Inner and outer products of tensors,
symmetrizations, and derivatives can all be specified as part of the
tensor definitions. Furthermore, index symmetries of the newly defined
tensor can be included. A calculation algorithm for the new tensor is
created automatically based on the definition and will automatically
take index symmetries into account in order to provide an efficient
calculation.

This booklet outlines the usage and various options available with the
\grcmd{grdef} command.\footnote{The \grcmd{grdef} command replaces the
\grcmd{grdefine} command which was the GRTensorIII standard for
definitions of new tensors from Versions 1.00 to 1.21. The
\grcmd{grdefine} command is still included for purposes of backwards
compatibility. See \texttt{?grdefine} for details of its use.}
%
%------------------------------------------------------------------------------
\section{Command syntax}
%------------------------------------------------------------------------------
%
There are two ways to define a new tensor using \grcmd{grdef}. The
first method is to simply state the name of the tensor, including its
`index structure' (ie. a list of covariant and contravariant
indices). The second method provides a complete definition in terms of
previously defined tensors.

Compare the following two invocations of \grcmd{grdef}:\footnote{The
specifics of the notation used by these two commands are described in
the sections to follow.}\\

\noindent\texttt{> grdef ( `A\{a b\}` ):}\\

\noindent\texttt{> grdef ( `G2\{a b\} := R\{a b\} - (1/2)*Ricciscalar*g\{a b\} 
                                   + lambda*g\{a b\}` ):}\\

The first form of the command defines a tensor whose component values
are arbitrary and can be manually specified for the background
spacetime. Specifically, it creates a definition of a covariant
two-index object, \texttt{A}. The indices are listed in curly braces,
\texttt{\{\}}, and assigned the labels \texttt{a} and \texttt{b}. This
tensor would be accessed as \texttt{A(dn,dn)} in commands such as
\grcmd{grcalc} and \grcmd{grdisplay}. Note, however, that although
\grcmd{grdef} creates a definition of the tensor, it does not
calculate its components automatically. The command\\

\noindent\texttt{> grcalc ( A(dn,dn) ):}\\

\noindent must be given in order to explicitly request the calculation
of the components for in current background spacetime.\footnote{See
Booklet \grCalcRef.} In this case, since no prescription for
calculating the components of this tensor has been given, when the
\grcmd{grcalc} command is issued the user is prompted to input the
components of \texttt{A(dn,dn)} manually.\\

The second version of \grcmd{grdef} listed above is somewhat more
complicated. In this case the command once again defines a
contravariant two-index tensor, \texttt{G2}, but this time it is
explicitly assigned to an expression involving a number of previously
defined tensors. The \grcmd{grdef} command uses this information to
automatically create a calculation function for the object
\texttt{G2(dn,dn)}. Thus, the command\\

\noindent\texttt{> grcalc ( G2(dn,dn) ):}\\

\noindent can be used to calculate the newly defined tensor based on its
given definition involving \texttt{R(dn,dn)}, \texttt{Ricciscalar} and 
\texttt{g(dn,dn)}.\\

The argument to \grcmd{grdef} which describes the definition of a new
tensor is a MapleV string (that is, enclosed in
back-quotes,~`\texttt{`}'). The string specifies a tensor equation
which includes information about the index structure, symmetries, and
summations, which are to be carried out in the calculation of the
tensor. This information is parsed by the \grcmd{grdef} command in
order to create a calculation function corresponding to the
definition. The \grcmd{grcalc} command uses this calculation function
in order to arrive at the components of the newly defined object in a
given background geometry. The next section of this booklet details the
format of the tensor definition string.\\

The syntax for the \grcmd{grdef} command is as follows:\\
\begin{cmdspec}
  \label{spec:grdef}
  \grcmdline{grdef ( \grarg{defString}, [\grarg{symSet}], [\grarg{asymSet}],
    [\grarg{rsumSet}] )}

  \begin{description}
    \item[\grarg{defString}] -- The definition string (described in
      Section~\ref{sec:defStr}).
    \item[\grarg{symSet}] -- (\textit{optional}) A set of lists of indices
      which are symmetric under interchange (described in 
      Section~\ref{sec:symList}).
    \item[\grarg{asymSet}] -- (\textit{optional}) A set of lists of indices
      which are anti-symmetric under interchange (described in
      Section~\ref{sec:symList}).
    \item[\grarg{rsumSet}] -- (\textit{optional}) A set of ranges over which
      dummy indices are to by summed (described in
      Section~\ref{sec:restrictedsum}).
  \end{description}

  \grexample{grdef ( `G2\{(a b)\} := G\{a b\} + lambda*g\{a b\}` )}
\end{cmdspec}

The next sections give details of the use of each of the arguments
listed above. However, the first-time reader may find it instructive
to skip ahead to the Examples section (Section \ref{sec:examples} of
this booklet) to get some familiarity with the syntax of the
\grcmd{grdef} command before reading the more in-depth descriptions
that now follow.
%
%-----------------------------------------------------------------------------
\section{Tensors and tensor expressions}\label{sec:defStr}
%-----------------------------------------------------------------------------
%
The main job of the \grcmd{grdef} command is to parse a string containing
a tensor definition. As an example, consider the argument of the second
example given above,
\begin{center}
  \texttt{`G2\{a b\} := R\{a b\} - (1/2)*Ricciscalar*g\{a b\} +
    lambda*g\{a b\}` }
\end{center}
This is a special case of the general form used for tensor assignments,
\begin{center}
  \texttt{`\grarg{newObject} := \grarg{objectDef}`}.
\end{center}
The string is interpreted by \grcmd{grdef} as an assignment of an
expression involving indexed objects, \grarg{objectDef}, to a new
indexed object, \grarg{newObject}. The format of this defining
expression is outlined in the following paragraphs.\footnote{As noted
in the previous section, the \grarg{objectDef} expression does not
need to be specified. In this case, the tensor is created without
assigning it a definition, and the components must be manually entered
using the command \grcmd{grcalc}.}\\

\noindent\textbf{Indexed objects:} 
Indexed objects (tensors and the like) are specified by
their name and index structure:
\begin{center}
	\texttt{\grarg{Name} \{ \grarg{indices} \}}
\end{center}
The \grarg{Name} is simply an unassigned MapleV name by which the
object is referenced, such as \texttt{R} in the case of the Riemann
tensor and its contractions, and \texttt{Chr} in the case of the
Christoffel symbols.\footnote{See Booklet \grCalcRef~for a list of
objects in the standard GRTensorIII library.}\\

The index list, \grarg{indices}, is always enclosed in curly braces,
\{\}.  Individual indices take the form of alphabetic characters
separated by spaces. Different classes of index are specifiable:
\begin{center}
  \begin{tabular}{rl}
    \texttt{a}, \texttt{b}, \texttt{c}, $\ldots$ & -- covariant indices,\\
    \texttt{\^{}a}, \texttt{\^{}b}, \texttt{\^{}c}, $\ldots$ & -- contravariant
      indices, \\
    \texttt{(a)}, \texttt{(b)}, \texttt{(c)}, $\ldots$ & -- covariant basis
      indices, \\
    \texttt{\^{}(a)}, \texttt{\^{}(b)}, \texttt{\^{}(c)}, $\ldots$ &
      -- contravariant basis indices.
  \end{tabular}
\end{center}
Thus, within the \grcmd{grdef} command, the Riemann tensor $R^a{}_{bcd}$ is
referenced using:
\begin{center}
  \texttt{R\{\^{}a b c d\} }
\end{center}
while the basis components of the same tensor are
\begin{center}
  \texttt{R\{\^{}(a) (b) (c) (d)\} }
\end{center}
The characters used in the index labels (\texttt{a}, \texttt{b},
$\ldots$) should be unassigned MapleV names. For instance, in the
above example, none of \texttt{a}, \texttt{b}, \texttt{c}, \texttt{d},
can have been assigned a value. This notation differs from that used
by such commands as \grcmd{grcalc} which uses the generic labels
`\texttt{up}', `\texttt{dn}', etc., to indicate index positions. The
\grcmd{grdef} notation attaches a unique label to each index. This
additional information is required to be able to specify index
summations and correspondences, as described below. (Scalars are
referenced using simply their name without an index list or curly
braces.)\\

\noindent\textbf{Derivatives of tensors:} The special characters
`\texttt{,}' (comma) and `\texttt{;}' (semi-colon) can be used to
specify partial derivatives and covariant derivatives in the index
lists of tensors. Thus the tensor $R_{ab;c}$ ($= \nabla_c R_{ab}$)
would be represented as:
\begin{center}
  \texttt{R\{a b ;c\}}
\end{center}

The specific action of the derivative operation is controlled by the
form of the index (tensor or basis) which immediately follows the
derivative indicator, as in the following examples:

\renewcommand{\baselinestretch}{1.5}\normalsize
\begin{center}
  \begin{tabular}{rl}
    \texttt{R\{\^{}a b ,c\}}:
      & $R^a{}_{b,c} := \frac{\partial R_{ab}}{\partial x^c}$ \\
    \texttt{R\{\^{}a b ;c\}}:
      & $R^a{}_{b;c} := R^a{}_{b,c} - \Gamma^d_{bc} R^a{}_d + 
        \Gamma^a_{bd} R^d{}_c$ \\
    \texttt{R\{\^{}(a) (b) ,(c)\}}:
      & $R^{(a)}{}_{(b),(c)} := \frac{\partial R^{(a)}{}_{(b)}}{\partial x^d}
        e_{(c)}{}^d$ \\
    \texttt{R\{\^{}(a) (b) ;(c)\}}:
      & $R^{(a)}{}_{(b);(c)} := R^{(a)}{}_{(b),(c)}
        - \gamma^{(d)}{}_{(b)(c)} R^{(a)}{}_{(d)}
        + \gamma^{(a)}{}_{(b)(d)} R^{(d)}{}_{(c)}$ \\
  \end{tabular}
\end{center}
\renewcommand{\baselinestretch}{1.0}\normalsize
These correspond to the actions of the index labels \texttt{pdn},
\texttt{cdn}, \texttt{pbdn}, and \texttt{cbdn} in commands such as
\grcmd{grcalc} (see Booklet \grCalcRef).\\

\noindent\textbf{Tensor sums and products:} The exterior product of a pair
of tensors is indicated by the `\texttt{*}' symbol. It is also
possible to specify the product of a scalar function and a tensor
using the same operator. Addition and subtraction of tensors is carried
out as usual by the `\texttt{+}' and `\texttt{-}' operators.  Round
braces, \texttt{()}, can be used to group terms. Thus, the tensor
expression
\[
  \frac{1}{2}\left( R_{abcd} + f(x) g_{ab} g_{cd} \right)
\]
would be represented as:

\begin{center}
  \texttt{ (1/2) * ( R\{a b c d\} + f(x)*g\{a b\}*g\{c d\} ) }
\end{center}

The index names and configurations of each term must be 
consistent with each other and with the left-hand side of a tensor
assignment.\\

\noindent\textbf{Inner products (summations over indices):} A summation
over the range of an index can be specified by repeating the index
name, once in the covariant and once in the contravariant position.
Thus, the expressions
\[
  \sum_b R^{ab}{}_{bc} \qquad\text{and}\qquad \sum_b R_a{}^b R_{bc}
\]
would be represented, respectively, as

\begin{center}
  \texttt{R\{\^{}a \^{}b b c\}}, $\qquad$ and $\qquad$ 
  \texttt{R\{a \^{}b\}*R\{b c\}}.
\end{center}

By default the summation is carried out over all index values, however
restricted summations (say over three dimensions out of four) can also
be specified, as described in Section \ref{sec:restrictedsum},
below.\\

\noindent\textbf{Operators:} Operators can be used in \grcmd{grdef}
just as they are used in calculations using \grcmd{grcalc}. The
argument of the operator is placed in square braces. Thus, the
command\\

\noindent\texttt{> grdef ( `X := R\{\^{}a \^{}b\}*Box[ R\{a b\} ]` ):}\\

\noindent serves to define the scalar
\[
  X := R^{ab}\Box R_{ab},
\]
A limitation on the use of operators, however, is that they can only
be used on individual GRTensorIII objects, not functions of objects.
To define, for instance, the object
\[
  T_{ab} := \Box (R_{acdb}R^{cd}),
\]
a two stage definition is needed. First an intermediate 2-tensor 
corresponding to the argument of the operator,
in this case $R_{acdb}R^{cd}$, is defined:\\

\noindent\texttt{> grdef ( `T1\{a b\} := R\{a c d b\}*R\{\^{}c \^{}d\}` ):}\\

\noindent The final object, $T_{ab}$, is then defined as an operator acting on
$T1_{ab}$:\\

\noindent\texttt{> grdef ( `T\{a b\} := Box[T1\{a b\}]` ):}\\

Nested operators, to define objects such as
\[
  \Box\Box R_{abcd},
\]
are handled in a similar fashion, defining intermediate
objects which correspond to the action of an operator on a single
object and building upon these.

New operators can not be defined using \grcmd{grdef}.\\

\noindent\textbf{Symmetrizations:} A short-hand notation can be used to 
indicate symmetrizations over a set of indices. The notation
corresponds to the common usage of round and square braces
to denote symmetrization and anti-symmetrization, respectively.
Specifically, define
\begin{align*}
  T_{abc \ldots (s_1 \ldots s_m) \ldots de} & 
    := \frac{1}{m!} \sum_\pi 
    T_{abc\ldots s_{\pi(1)} \ldots s_{\pi(m)} \ldots de},\\
  T_{abc \ldots [s_1 \ldots s_m] \ldots de} &
    := \frac{1}{m!} \sum_\pi (-1)^{\text{sign}(\pi)}
    T_{abc \ldots s_{\pi(1)} \ldots s_{\pi(m)} \ldots de},
\end{align*}
where the summations are taken over all permutations, $\pi$, of the
numbers $1\ldots m$, and $\text{sign}(\pi)$ represents the sign (odd or
even) of the permutation.\\

The tensor
\[
  T_{a(bcd)} := \frac{1}{6}\left( T_{abcd} + T_{acdb} + T_{adbc} +
    T_{abdc} + T_{adcb} + T_{bdc} \right), 
\]
could be represented in \grcmd{grdef} by the string\footnote{A
technical point arises from the fact that round braces are used to
indicate both symmetrizations as well as basis indices. However, it is
always possible for the parser to determine from the context which is
intended. If the braces surround a single index, then the index is
regarded as a basis index, otherwise a symmetrization is
assumed.}${}^,$\footnote{Spaces are not needed between indices where
braces occur. The example given here could also be entered as
\texttt{T\{a(b c d)\}}, omitting the space between the \texttt{a} and
\texttt{b}.}
\begin{center}
  \texttt{T\{a (b c d)\}},
\end{center}
while the tensor
\[
  T_{a[bcd]} := \frac{1}{6} \left(
   T_{abcd} + T_{acdb} + T_{adbc} - T_{abdc} - T_{adcb} - T_{bdc} \right),
\]
could be represented by
\begin{center}
  \texttt{T\{ a [b c d]\}}.
\end{center}

Symmetrizations can also be carried over tensor products. Thus
\[
  R^a{}_{bc(d}R_{e)f}
\]
could be written
\begin{center}
  \texttt{R\{\^{}a b c (d\}*R\{e) f\}}
\end{center}

Braces are determined to be `covariant' or `contravariant' based on
the index which immediately follows an open-brace, \texttt{(},
\texttt{[}, and the index which precedes a close-brace, \texttt{)},
\texttt{]}. Braces enclose only indices of the same character. Thus
the name
\begin{center}
  \texttt{ T\{(a \^{}b c)\} }
\end{center}
indicates that the symmetrization is to be performed only over the
covariant indices within the round braces, ie. \texttt{a} and \texttt{c}.

Finally, indices can be excluded from symmetrizations by
placing them between vertical bars, `\texttt{|}'. Thus if a symmetrization
were to be carried out over the second and fourth index of a four index
tensor, this could be written as
\[
  T_{a(b|c|d)} := \frac{1}{2}\left( T_{abcd} + T_{adcb} \right),
\]
which in \grcmd{grdef} notation is
\begin{center}
  \texttt{T\{a (b | c | d)\}}.
\end{center}
%
%------------------------------------------------------------------------------
\section{Index symmetries}
%------------------------------------------------------------------------------
%
Symmetries among tensor indices can be used in calculation programs to
significantly reduce the time taken for a calculation by recognizing
redundant components. The \grcmd{grdef} command can not determine on
its own when such index symmetries exist. However, symmetries can be
listed explicitly in the tensor definition so that the resulting
calculation function is as efficient as possible.\\

The notation used to indicate symmetric and anti-symmetric index
groups is identical to that used for symmetrizations, described in the
previous section. Thus if braces are placed within the index list of
the \textit{new} tensor (on the left-hand side of the tensor
assignment) these are interpreted as index symmetries rather than
symmetrizations.\\

To give some explicit examples, round braces, \texttt{()}, indicate that
a tensor is symmetric in the enclosed indices. The assignment\\

\noindent\texttt{> grdef ( `A\{(a b)\} := R\{a b\}` ):}\\

\noindent assigns a calculation function to the tensor
$A_{ab}$ which assumes that it is symmetric in its indices, ie.,
\[
  A_{ab} = A_{ba}.
\]
The calculation algorithm called by the command \\

\noindent\texttt{> grcalc ( A(dn,dn) ):}\\

\noindent then calculates only the components in the upper diagonal
and cross-assigns the components in the lower diagonal. By the same
token, the command\\

\noindent\texttt{> grdef ( `A\{(a b)(c d)\}` ):}\\

\noindent assigns to the object \texttt{A(dn,dn,dn,dn)} a calculation
algorithm which assumed the tensor is symmetric in its first two and
last two indices,\footnote{A calculation function satisfying the
required symmetries is created automatically if such a function does
not already exist among the existing GRTensorIII object definitions.}
\[
  A_{abcd} = A_{bacd} = A_{abdc} = A_{badc}.
\]

\indent Since in this case $A_{abcd}$ was not given a defining
expression, a call to \grcmd{grcalc} would prompt the user to enter
its components explicitly. However, it would only be necessary to
input the independent components (up to the specified index
symmetries).\\

Square braces act similarly, indicating indices in which the newly
defined object is anti-symmetric. Thus\\

\noindent\texttt{> grdef ( `A\{[a b c] d\}` ): }\\

\noindent defines a tensor for which it is assumed that
\[
  A_{abcd} = A_{bcad} = A_{cabd} = -A_{acbd} = -A_{bacd} = -A_{cbad}.
\]

The `covariant' or `contravariant' character of the braces is
determined as for symmetrizations, described in the previous
section. Also as above, indices which are to be excluded from a
symmetrization can be enclosed in vertical bars, `\texttt{|}'.\\

Note that the symmetries need to be stated explicitly on the left-hand
side of the assignment. Although in the definition of $A_{ab}$ above
the index symmetry is obvious from the properties of the Ricci tensor,
the \grcmd{grdef} command does not search for such properties. It must
be instructed as to which index symmetries to include in the
calculation function for the tensor.

Another important consequence of this is that the symmetries which are
stated explicitly will be incorporated into the calculation function
regardless of whether the listed symmetry is an actual property of
the tensor which is being defined. Thus the definition\\

\noindent\texttt{> grdef ( `A\{[a b]\} := R\{\^{}c a c b\}` ):}\\

\noindent will assign a calculation function to $A_{ab}$ which assumes
that the tensor is anti-symmetric, which it clearly is not. If the
\grcmd{grcalc} command is subsequently used to calculate $A_{ab}$ for
a particular metric, it will always arrive at the result
\[
  A_{ab} = 0,
\]
which is incorrect in general.\\

Finally, the different role of braces on the left-hand side and
right-hand side of a tensor assignment are emphasized: On the
left-hand side, braces are informational, describing the symmetries
of the newly defined object. On the right-hand side, the
braces specify that the operation of symmetrization is to be carried
out over the enclosed indices. Thus the command\\

\noindent\texttt{> grdef ( `A\{(a b)\} := B\{(a b)\}` ):}\\

\noindent defines the tensor
\[
  A_{ab} := \frac{1}{2} \left( B_{ab} + B_{ba} \right)
\]
(as indicated by the \texttt{()}-braces on the right-hand side), which
is assumed to be symmetric in its indices (as indicated by the
\texttt{()}-braces on the left-hand side). If the braces on the left
of the \grcmd{grdef} command are omitted, the calculation function
assigned to the new object would not take into account the symmetry
inherent in $A_{ab}$, and, for example, the components $A_{12}$ and
$A_{21}$ would be calculated independently, even though by examining the
right-hand side of the definition it is clear that they are equal.
%
%------------------------------------------------------------------------------
\subsection*{Alternate specification of index symmetries}\label{sec:symList}
%------------------------------------------------------------------------------
%
The notation used to specify tensor indices in \grcmd{grdef} has been
chosen to mirror as closely as possible the standard notation of
textbooks of classical tensor analysis. However, because the text
typed onto an input line is linear (lacking visual super-scripts and
sub-scripts) the presence of a large number of braces among covariant
and contravariant indices can serve to obscure the contents of an
index list. Not only can this make the worksheet difficult to read
later on, it also increases the chance of an error being introduced
into the tensor definition.

To alleviate this problem somewhat, an alternate method exists for
specifying symmetries among the tensor indices. The \grcmd{grdef}
command possesses optional extra arguments which can be used to enter
symmetry and anti-symmetry lists. This can be used instead of the
placing of braces among the tensor indices in the left-hand side of
the definition string.\\

In these optional arguments, symmetries are expressed as lists of
index numbers: The list \texttt{[2,3,5]} indicates that the second,
third and fifth indices are involved in a symmetry (or
anti-symmetry). To specify symmetries, include the argument
\begin{center}
  \texttt{sym=\{\grarg{symLists}\}}
\end{center}
to the \grcmd{grdef} command. Thus,\\

\noindent\texttt{> grdef ( `A\{a b c d\}`, sym=\{[1,2],[3,4]\} ):} \\

\noindent indicates that the new tensor $A_{abcd}$ should be considered
to be symmetric in its first two and last two indices:
\[
  A_{abcd} = A_{bacd} = A_{abdc} = A_{badc}.
\]
(Note that this invocation of \grcmd{grdef} has an identical effect as
one of the examples of the previous section.) Alternatively,\\

\noindent\texttt{> grdef ( `A\{a b c d\}`, sym=\{[2,3,4]\} ):}\\

\noindent defines a tensor which is symmetric under interchange of its
second third and fourth indices:
\[
  A_{abcd} = A_{acdb} = A_{adbc} = A_{abdc} = A_{acbd} = A_{adcb}.
\]

Anti-symmetries are specified the same way using the argument
\begin{center}
  \texttt{asym=\{\grarg{asymLists}\} }
\end{center}
The tensor defined
by\\

\noindent\texttt{> grdef ( `A\{a b c\}`, asym=\{[1,2]\} ):}\\

\noindent is assigned a calculation function which assumes it is
anti-symmetric in its first two indices,
\[
  A_{abc} = -A_{bac},
\]
while the command\\

\noindent\texttt{> grdef ( `A\{a b c d\}`, sym=\{[1,3]\}, 
asym=\{[2,4]\} ):}\\

\noindent assigns a calculation function which is symmetric in indices
1 and 3 and anti-symmetric in indices 2 and 4.\\

Finally, note that a limitation of this notation (and the use of braces
as described in the previous section) is that some more complicated index
symmetries such as those involving groups of indices, can not be
specified. An important example is the index pair symmetry of the
Riemann tensor,
\[
  R_{abcd} = R_{cdab}.
\]
Although \grcmd{grdef} can not currently automatically create such
symmetry functions, the Riemann symmetry function has been pre-programmed,
and can be assigned to newly defined objects using the optional argument
\texttt{symfn}. This argument has the form
\begin{center}
  \texttt{symfn = \grarg{objectName}}
\end{center}
where \grarg{objectName} is the name of a pre-defined object with the
same symmetries as the object to be newly defined. Thus a new object
with Riemann index symmetries can be created using the definition\\

\noindent\texttt{> grdef ( `T\{a b c d\}`, symfn=R(dn,dn,dn,dn) ):}\\

\noindent Symmetry functions assigned in this way overide any other
index symmetry specifications (using braces or the \texttt{sym=} or
\texttt{asym=} parameters) for the newly defined object.
%
%------------------------------------------------------------------------------
\section{Special cases}
%------------------------------------------------------------------------------
%
%------------------------------------------------------------------------------
\subsection*{Defining vectors}
%------------------------------------------------------------------------------
%
Single index objects form a special case within \grcmd{grdef}. They can
be defined as usual by the means given in the previous section. Alternatively,
however, they can also be assigned a value directly by specifying the
components as a list. For example, the command\\

\noindent\texttt{> grdef ( `v{\^{}a} := [0,0,0,1]` ):}\\

\noindent defines a single index object, \texttt{v}, whose four components
are assigned the values \texttt{[0,0,0,1]}. This form of the
\grcmd{grdef} command alleviates the need to run \grcmd{grcalc} after
the definition in order to assign values to the components. However,
if the default spacetime is changed (either by creating or loading a
new background\footnote{See Booklet \grMakegRef.}), then
\grcmd{grcalc} would still be required in order to assign the
components of the vector in the new geometry.
%
%------------------------------------------------------------------------------
\subsection*{The Kronecker delta}
%------------------------------------------------------------------------------
%
A special object, \texttt{kdelta\{\^{}a b\}}, allows single components
of tensors to be isolated in tensor definitions. For example, if the
coordinates of the background spacetime are $(r,theta,phi,t)$, the
$t$-component can be selected using the object\\
\begin{center}
  \texttt{kdelta\{\^{}a \$t\}}.\\
\end{center}
The `\$' character preceding the coordinate name $t$ indicates that that
the \texttt{kdelta} object is to take the form
\[
  \delta^a{}_t :=
    \left\{
      \begin{array}{l}
        1, \quad\text{if}\quad a=t,\\
        0, \quad\text{otherwise}.
      \end{array}
    \right.
\]
Thus a vector in the $t$-direction could be defined using the
command\\

\noindent\texttt{> grdef ( `v\{\^{}a\} := f(t)*kdelta\{\^{}a \$t\}` ):}\\

\noindent (which is equivalent to the command\\

\noindent\texttt{> grdef ( `v\{\^{}a\} := [0,0,0,f(t)]` ):}\\

\noindent in the given coordinates). Another example is a 2-index tensor,
\[
  T_{ab} := P(r,t) g_{ab} + (P(r,t) + \rho(r,t)) \delta^t{}_a \delta^t{}_b,
\]
which could be defined using the command\\

\noindent\texttt{> grdef ( `T\{(a b)\} := }\\
\indent\texttt{\hspace{10mm} P(r,t)*g\{a b\}
  + ( P(r,t) + rho(r,t) )*kdelta\{a \$t\}*kdelta\{b \$t\}` ):}
%
%------------------------------------------------------------------------------
\subsection*{Multiple metrics}
%------------------------------------------------------------------------------
%
The \grcmd{grdef} command allows the definition of objects which depend
on multiple background geometries. Such objects arise, for instance, when
one considers the junction between two spacetimes described by different
metrics. In a tensor definition, objects which are to use alternate
background geometries are indexed using angle braces, \texttt{<>}. For
example, the following definition can be used to define a tensor,
\texttt{Dg}, which is the difference of two metrics:\\

\noindent\texttt{> grdef ( `Dg\{a b\} := g<1>\{a b\} - g<2>\{a b\}` ):} \\

\noindent In this definition, the indices \texttt{<1>} and \texttt{<2>}
in the objects \texttt{g\{a b\}} indicate that the components of
\texttt{g(dn,dn)} should be taken from metrics specified by the user
at calculation time. To calculate the object \texttt{Dg}, the command\\

\noindent\texttt{> grcalc ( 1=ssym, 2=schw, Dg(dn,dn) ):}\\

\noindent would be used, where in this example \texttt{ssym} and 
\texttt{schw} are the names of spacetimes that have been
previously loaded in the session.  In the calculation of \texttt{Dg},
the objects in its definition that are labeled with the index
\texttt{<1>} would be taken from the \texttt{ssym} spacetime, while
objects labeled by \texttt{<2>} would use components from the
\texttt{schw} spacetime.
%
% DISABLE
\iffalse
%------------------------------------------------------------------------------
\subsection*{Restricted summations}\label{sec:restrictedsum}
%------------------------------------------------------------------------------
%
A further optional argument to the \grcmd{grdef} command allows
summations over indices to be restricted to a certain range of index
values.  Consider the definition of the scalar
\[
  \phi := T_{ab} u^a u^b,
\]
where, for some reason, we wish to restrict the implied summation over the
dummy indices $a$ and $b$ to the range $2\ldots 4$. The \grcmd{grdef}
command creating such a definition is:\\

\noindent{\texttt{> grdef ( `phi := T\{a b\}*u\{\^{}a\}*u\{\^{}b\}`,
  restrict=\{a=2..4, b=2..4\} ):}\\

\noindent The \texttt{restrict=\{\}} argument is a set specifying the
ranges over which summations over dummy indices are to be performed.
Any indices not listed in the set are assumed to be summed over their
entire range.
\fi
% RE-ENABLE
%
%------------------------------------------------------------------------------
\section{Examples}\label{sec:examples}
%------------------------------------------------------------------------------
%
In this section, a number of \grcmd{grdef} commands are listed,
summarizing the use of the command. A number of the following examples
are mirrored from the previous sections where they are described
in more detail. The \grcmd{grdef} command is also used in a number
of the demonstration worksheets available from the GRTensorII
world-wide-web site, \texttt{http://astro.queensu.ca/\~{}grtensor/}.

\begin{enumerate}
  \item A generic contravariant single indexed object (vector):\\

    \texttt{> grdef ( `u\{\^{}a\}` ):}\\

  \item A generic covariant single indexed object (1-form):\\

    \texttt{> grdef ( `v\{a\}` ):}\\

  \item A generic 2-index tensor, anti-symmetric in its indices:\\

    \texttt{> grdef ( `A\{[a b]\}` ):}\\

  \item The covariant derivative of the Weyl tensor, $C_{abcd;e}$:\\

    \texttt{> grdef ( `dC\{[a b] [c d] e\} := C\{a b c d ;e\}` ):}\\

  \item The component of the Ricci tensor measured along a particular vector, 
    $\phi := R_{ab} u^a u^b$:\\

    \texttt{> grdef ( `phi := R\{a b\}*u\{\^{}a\}*u\{\^{}b\}` ):}\\

  \item The Einstein tensor with cosmological constant:\\

    \texttt{> grdef ( `G2\{(a b)\} := G\{a b\} + lambda*g\{a b\}` ):}\\

    (Note the use of braces on the left-hand side to indicate the index 
    symmetry. Although the tensors $g_{ab}$ and $G_{ab}$ have a clear
    index symmetry, the \grcmd{grdef} command will only apply symmetries
    explicitly stated on the left-hand side of the definition.)

  \item Basis components of the Einstein tensor with cosmological constant:\\

    \texttt{> grdef ( `G2\{((a) (b))\} := G\{(a) (b)\} +
      lambda*eta\{(a) (b)\}` ):}\\

    (This command is redundant if the previous definition of \texttt{G2(dn,dn)}
    is already created, since the basis components could be calculated
    automatically using \texttt{grcalc ( G2(bdn,bdn) )}. See also the note
    at the end of this section.)

  \item The Bel-Robinson tensor,
    $T_{abcd} := C_{aecf} C_b{}^e{}_d{}^f -
    \frac{3}{2} g_{a[b} C_{jk]cf} C^{jk}{}_d{}^f$.\\

    \texttt{> grdef ( `T\{(abcd)\} := C\{a e c f\}*C\{b \^{}e d \^{}f\}
      - (3/2)*g\{a [b\}*C\{j k] c f\}*C\{\^{}j \^{}k d \^{}f\}` ):}\\

    (An alternate definition of the Bel-Robinson tensor is an example
    in Booklet \grIntroRef.)

  \item A timelike vector field, $v^a := [0,0,0,f(r,t)]$:\\

    \texttt{> grdef ( `v\{\^{}a\} := [0,0,0,f(r,t)]` ):}\\

  $\ldots$ and a corresponding perfect fluid energy-momentum tensor:\\

    \texttt{> grdef ( `T\{(a b)\} := P(r,t)*g\{a b\}
      + (rho(r,t) + P(r,t))*v\{a\}*v\{b\}` ):}\\

  \item An object with Riemann symmetries:\\

    \texttt{> grdef ( `T\{a b c d\} := (1/6)*g\{a[c\}*g\{d]b\}*Ricciscalar`,
      symfn=R(dn,dn,dn,dn) ):}
\end{enumerate}

\noindent Object names can not be used twice. If an object with the
same name already exists (either in the standard library or as
a result of a previous use of \grcmd{grdef}), its definition must
be removed before it can be re-used. The \grcmd{grundefine} command
performs this task:\\
%
\begin{cmdspec}
  \label{spec:grundefine}
  \grcmdline{grundefine ( \grarg{objectName} )}

  \begin{description}
    \item[\grarg{objectSeq}] -- A sequence objects whose definitions are
      to be removed.
  \end{description}

  \grexample{grundefine ( G(dn,dn) )}
\end{cmdspec}
%
% DISABLE NEXT SECTIONS!
\iffalse
%------------------------------------------------------------------------------
\section{Saving and loading definitions}
%------------------------------------------------------------------------------
%
Definitions created with \grcmd{grdef} can be saved to a file using the
\grcmd{grsavedef} command, allowing them to be used in other sessions and
applications. In this way, collections of objects not in the standard
GRTensorIII library can be maintained.\\
\begin{cmdspec}
  \label{spec:grsavedef}
  \grcmdline{grsavedef ( \grarg{objectSeq}, \grarg{fileName} )}

  \begin{description}
    \item[\grarg{objectSeq}] -- A sequence of objects whose definitions
      are to be saved to a file.
    \item[\grarg{fileName}] -- The name of a file
      where the object definitions are to be saved.
  \end{description}

  \grexample{grsavedef ( G2(dn,dn), T(dn,dn,dn,dn), `newdefs.mpl` )}
\end{cmdspec}

For instance, the Einstein and Bel-Robinson tensors defined in the
previous section could be saved to the file `\texttt{newdefs.mpl}'
using the command:\\

\noindent\texttt{> grsavedef ( G2(dn,dn), T(dn,dn,dn,dn), `newdefs.mpl` ):}\\

\noindent If the newly created file is to reside in a directory other than
the one in which the MapleV session was started, then the full path
name must be used. Note also that \grcmd{grsavedef} makes use of the
MapleV \grcmd{write} command, which does not check for the existence
of the file which is being written. Thus if a file of the same name
already exists in the specified directory \textit{it will be
overwritten}.

Files created by the \grcmd{grsavedef} command can be read into a
new GRTensorIII session using the command \grcmd{grloaddef}:\\
\begin{cmdspec}
  \label{spec:grloaddef}
  \grcmdline{grloaddef ( \grarg{fileName} )}

  \begin{description}
    \item[\grarg{fileName}] -- The name of the file (containing definitions
      created by \grcmd{grsavedef}) to be loaded.
  \end{description}

  \grexample{grloaddef ( `newdefs.mpl` )}
\end{cmdspec}
\fi
% END OF DISABLE
%
%------------------------------------------------------------------------------
%\vfill
% No citations in this doc!
%\bibliographystyle{unsrt}
%\bibliography{grtensor}
%------------------------------------------------------------------------------
\pagebreak
\section*{Commands described in this booklet:}
  \noindent
  \grcmdline{grdef ( \grarg{defString}, [\grarg{symSet}], [\grarg{asymSet}], 
    [\grarg{rsumSet}] )}
   \dotfill \pageref{spec:grdef}\\

  \noindent
  \grcmdline{grundefine ( \grarg{objectName} )}
  \dotfill \pageref{spec:grundefine}\\

%  \noindent
 % \grcmdline{grsavedef ( \grarg{objectSeq}, \grarg{fileName} )}
 %\dotfill \pageref{spec:grsavedef}\\
%
% \noindent
% \grcmdline{grloaddef ( \grarg{fileName} )}
%\dotfill \pageref{spec:grloaddef}
%
%------------------------------------------------------------------------------
\vfill
\large
\noindent The information contained in this booklet is also available from the
following online help pages:\\

\noindent\texttt{?grdef}, \texttt{?grundefine}, \texttt{?grsavedef},
\texttt{?grloaddef}, \texttt{?kdelta}, \texttt{?grt\_objects},
\texttt{?grt\_basis}, \texttt{?grt\_operators}, \texttt{?grcalc},
\texttt{?grdefine}.
%
%------------------------------------------------------------------------------
\end{document}
%==============================================================================
