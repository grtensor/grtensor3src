%=================================================================-*-LaTeX-*-==
% GRTensorII 1.50 Manual
% Booklet F: Setup and defaults
%
% Denis Pollney
% July 1996
%==============================================================================
\documentclass{article}
\usepackage{amsmath}
\usepackage{amssymb}
%\usepackage{maple2e}
\usepackage{grtensor}
\usepackage{grbooks}
%\textheight 9in
%------------------------------------------------------------------------------
\begin{document}
\grlabel{\grSetupLabel}
\grtitle{\grSetupTitle}
\grdate{Nov 2016}
\grtitlepage
%==============================================================================
% Body.
%==============================================================================
\copyrightpage
This booklet outlines the installation and configurable options
of the GRTensorII package. 
%
%------------------------------------------------------------------------------
\section{Installation} \label{sec:install}
%------------------------------------------------------------------------------
%
The GRTensorII package is distributed as a set of MapleV library
files, as follows:
\begin{description}
  \item[\texttt{griii.m}] -- contains the GRTensorII program itself
    composed of the commands described in this series of booklets,
    along with the standard tensor object library.

  \item[\texttt{basislib.m}] -- contains definitions for objects defined
    for bases and NP tetrads.

  \item[\texttt{invar.m}] -- contains definitions for the scalar
    invariants of the Riemann tensor.

  \item[\texttt{dinvar.m}] -- contains definitions of some
    differential invariants (see \texttt{?dinvar}).

  \item[\texttt{grtools.m}] -- contains some extra commands (see
    \texttt{?difftool}, \texttt{?grlimit}).
\end{description}
With the MapleV.3 version of GRTensorII, there is also included
a library \texttt{griihelp.m} which contains the online help
information. The corresponding file for the MapleV.4 version
is called \texttt{maple.hdb}. (See Section \ref{sec:help}).\\

In addition to these libraries, a directory of input spacetimes
(a collection of \texttt{*.mpl} files) is also distributed
with GRTensorII.\\

The library files should be placed in a directory within the
MapleV library path specified by the variable \texttt{libname}.
To change the value of this variable so that it includes a
particular directory, use the command\\

\noindent\texttt{> libname := `c:/users/grtensor/lib/`, libname:}\\

\noindent which in this case appends the directory 
\texttt{c:/users/grtensor/lib} to the \texttt{libname} sequence. 
On Unix machines, by adding this command to a file named
\texttt{.mapleinit} in the user's home directory, it will be executed
whenever MapleV is started, ensuring that the library directory is
accessible\ (for MS-DOS/Windows based machines, the file
\texttt{maple.ini} serves the same
purpose).%\footnote{See also \texttt{?libname}.}\\
%
%\begin{figure}
\begin{cmdspec}
The installation procedure for GRTensorII is as follows:
  \begin{enumerate}
    %
    \item Copy the GRTensorII library files into their own directory or
      the standard MapleV library directory on the machine. (MapleV.4
      users should be careful not to overwrite the standard \texttt{maple.hdb}
      with GRTensorII's \texttt{maple.hdb} help file.)
    %
    \item Copy the GRTensorII metric files into their own directory.
    %
    \item Ensure that the directory in which the library files have been
      stored (Step 1) is listed in the MapleV \texttt{libname} variable.
      If not, then modify the \texttt{.mapleinit} file (on Unix platforms,
      otherwise \texttt{maple.ini}), as described below.
    %
    \item Create a file called \texttt{grtensor.ini} in one of the
      directories specified by the \texttt{libname} variable. To this
      file, add a line which assigns the variable
      \texttt{grOptionMetricPath} to the metric directory specified in
      Step 2.\footnote{The \texttt{grOptionMetricPath} variable is
      described in Section \ref{sec:groptions}. The
      \texttt{grtensor.ini} file is described in Section
      \ref{sec:init}.}
  \end{enumerate}
\end{cmdspec}
%\end{figure}

\noindent\textbf{Startup:} The command\\

\noindent\texttt{> readlib ( grii ):}\\

\noindent reads the GRTensorII library into MapleV. At this point
all of the GRTensorII commands described in these booklets are
available to the user. To initialize the program, run the command\\

\noindent\texttt{> grtensor():}\\

\noindent If a file called \texttt{grtensor.ini} exists in one of the
directories specified by the \texttt{libname} variable, it is read
and executed by the \grcmd{grtensor} command (see Section \ref{sec:init}).

To check that GRTensorII is installed properly, start a new MapleV
session and execute the following commands:\\

\noindent\texttt{> readlib ( grii ):\\
\noindent > grtensor():\\
\noindent > qload ( npschw ):\\
\noindent > grcalc ( RicciSc ):\\
\noindent > grdisplay ( \_ ):}\\

\noindent This set of commands checks that the \texttt{grii.m} library
is readable, loads a spacetime from the directory specified by
\texttt{grOptionMetricPath}, and carries out a simple calculation to
ensure that the automatically loading libraries (\texttt{basislib.m}
in this case) can be read properly.\\

\noindent\textbf{Support libraries:} For the MapleV.4 version, online
help will be read as it is requested from the \texttt{maple.hdb} file
included with GRTensorII, provided it is in the library path.
For the MapleV.3 version, however, the online help pages are
not read by default when the GRTensorII library is loaded,
but must be loaded separately using the command\\

\noindent\texttt{> readlib ( griihelp ):}\\

\noindent This command can also be added to the \texttt{grtensor.ini} file so
that it will be executed automatically when the GRTensorII package is
initialized by the \texttt{grtensor()} command.\\

The only other library which needs to be loaded manually is the
\texttt{grtools.m} library, which can be read at any time using
\texttt{readlib}. The other libraries (\texttt{basislib.m},
\texttt{invar.m}, \texttt{dinvar.m}) contain GRTensorII object
definitions which will be automatically loaded when they are needed
(by commands such as \grcmd{grcalc}%\footnote{See Booklet
%\grCalcRef.}
). They can also be loaded manually, however this should
\textit{not} be done using \grcmd{readlib}. Instead, the command
\grcmd{grlib} exists for the purpose of loading GRTensorII object
libraries:\\
\begin{cmdspec}
  \label{spec:grlib}
  \grcmdline{grlib ( \grarg{libName} )}

  \begin{description}
    \item[\grarg{libName}] -- The name of a GRTensorII object library
      which is to be loaded.
  \end{description}

  \grexample{grlib ( basislib )}
\end{cmdspec}
%
%------------------------------------------------------------------------------
\section{Initialization} \label{sec:init}
%------------------------------------------------------------------------------
%
The GRTensorII environment can be controlled by the creation of an
initialization file. In addition to displaying version information,
the command \grcmd{grtensor}\label{spec:grtensor} searches the MapleV
library path\footnote{Specified by the \texttt{libname} variable (see
\texttt{?libname}).} for the file \texttt{grtensor.ini}. This file can
contain any number of MapleV commands, and in particular GRTensorII
commands, to set up the environment in which GRTensorII is run. For
instance, the \texttt{grtensor.ini} file commonly contains settings
for the \texttt{grOption} variables (described in Section
\ref{sec:groptions}), such as directories in which to locate metric
files.

For example, if metrics are to be loaded from the 
\texttt{c:/users/grtensor/metrics} directory, the command\\

\noindent\texttt{grOptionMetricPath := `c:/users/grtensor/metrics/`:}\\

\noindent could be added to the \texttt{grtensor.ini} file. The
\grcmd{grtensor} command reads and executes the commands in the
initialization file, in this case setting the \texttt{grOptionMetricPath}
variable to its desired default value.
%
%------------------------------------------------------------------------------
\section{Online help} \label{sec:help}
%------------------------------------------------------------------------------
%
Online help is provided for all of the GRTensorII commands described in
these booklets. The method of providing the help differs slightly from
MapleV.3 to MapleV.4.

In the MapleV.3 version of GRTensorII, online help is available upon
loading the library \texttt{griihelp.m}. (The help screens have been
kept separate from the program library itself in order allow the user
the option of not including the help screens in situations where
memory usage is critical.) Online help can be loaded automatically
when the \grcmd{grtensor} command is executed by adding the line\\

\noindent\texttt{readlib ( griihelp ):}\\

\noindent to the \texttt{grtensor.ini} file.\\

For the MapleV.4 version, the online help is contained in the file
\texttt{maple.hdb} provided with the GRTensorII package. This file
should be located in a directory specified by the \texttt{libname}
variable. It will be read as help is requested.\\

Help pages exist for each of the GRTensorII commands described in
these booklets. In addition, the following pages provide some additional
summarized information:
\begin{description}
  \item[\texttt{?grt\_commands}] -- a summary of GRTensorII commands.
  \item[\texttt{?grt\_objects}] -- a summary of objects in the standard
    GRTensorII library. See Booklet \grCalcRef.  
  \item[\texttt{?grt\_operators}] -- a list of GRTensorII operators
    contained in the standard library. See Booklet \grCalcRef.
  \item[\texttt{?grt\_basis}] -- a list of objects contained in the
    standard libarary useful for calculations in spacetimes described
    by a basis or null tetrad.
\end{description}

\noindent Additional information can be found at the GRTensorII world wide web
site,
\begin{center}
  \texttt{http://astro.queensu.ca/\~{}grtensor/}
\end{center}
Inquiries and bug reports should be directed to
\begin{center}
  \makeatletter
  \texttt{grtensor@astro.queensu.ca}
  \makeatother
\end{center}
The authors welcome any and all feedback regarding the GRTensorII package
and its use.
%
%------------------------------------------------------------------------------
\section{Global option variables} \label{sec:groptions}
%------------------------------------------------------------------------------
%
A number of global variables control the GRTensorII environment. Their
values can be displayed using the command
\grcmd{groptions}\label{spec:groptions}. The values of the
\texttt{grOption} variables can be changed by re-assigning them. For
example,\\

\noindent\texttt{> grOptionCoordNames := false:}\\

\noindent sets the value of the \texttt{grOptionCoordNames} variable.
The following option variables can be set:

\begin{description}
  \item[\texttt{grOptionAlterSize}] (boolean, default=\texttt{false})
    If this variable is set to \texttt{true}, the new size of each
    component modified by \grcmd{gralter} is displayed as the command
    executes.  Sizes are given in MapleV words.

  \item[\texttt{grOptionCoordNames}] (boolean, default=\texttt{true})
    This variable controls whether tensor indices are displayed as
    names (eg. t, r, theta, phi) or numbers when their components
    are output using \grcmd{grdisplay}.

  \item[\texttt{grOptionDefaultSimp}] 
    (integer or name, default=\texttt{normal})
    This variable controls the type of simplification called during the
    calculation of an object by the \grcmd{grcalc} command. The default
    is to use the MapleV routine \grcmd{normal}. The
    \texttt{grOptionDefaultSimp} variable can be set to any of:
    \begin{center}
      \texttt{simplify, trig, power, hypergeom, radical, expand, factor,
      normal, sort, sqrt, trigsin, cons, consr}
    \end{center}
    The names correspond to the names used in the second argument
    of the \grcmd{gralter} command (see Booklet \grCalcRef).

    In general it is recommended that \grcmd{gralter} be used when
    simplification becomes necessary rather than changing the global
    simplification via this variable, as it is unlikely that any
    particular choice will work well in general situations, and an
    incorrect choice can greatly increase calculation times.

  \item[\texttt{grOptionDisplayLimit}] (integer, default=5000) This
    integer variable controls the size threshold beyond which a tensor
    component will not be displayed. If the component exceeds this
    limit only its size is displayed. The value is measured in words
    as determined by the MapleV \grcmd{length} function.

  \item[\texttt{grOptionLLSC}] (boolean, default=\texttt{true}) This
    variable indicates whether the Landau-Lifshitz spacelike signature
    convention is to be assumed on loading and defining new
    spacetimes.  If set to \texttt{true}, then four dimensional
    spacetimes specified by a metric, $g_{ab}$, or a general basis for
    which the signature has not been explicitly given, are assumed to
    have signature +2 (where signature is defined as the number of
    positive metric components minus the number of negative metric
    components in a locally orthonormal basis). Bases satisfying an NP
    inner product are assumed to have signature -2. If the
    \texttt{grOptionLLSC} variable is set to \texttt{false}, then the
    \texttt{sig} object will not be assigned unless it is explicitly
    given by the spacetime input file or specified in \grcmd{makeg}.

  \item[\texttt{grOptionMetricPath}] (string, default=homedir) This
    option variable is a MapleV string which specfies the default
    directory from which metric files are to be loaded (by the
    \grcmd{qload} command or to which they will be saved (by the
    \grcmd{makeg} command).  The sub-directory names should be
    separated using a forward slash, `\texttt{/}', even on MS-DOS/Windows
    systems. On these systems directory names must also end with
    a `\texttt{/}', as in
    \begin{center}
      \texttt{`c:/users/grtensor/metrics/`}
    \end{center}

  \item[\texttt{grOptionqloadPath}] (sequence, default=unassigned)
    This variable can be used to specify a set of directories which
    will be searched by \grcmd{qload} when loading metrics. Individual
    directory names are listed as MapleV strings with sub-directories
    separated by the forward slash character, `\texttt{/}'. Directores
    are searched in the order in which they are listed. If this
    variable is left unassigned, or if the metric file is not found in
    any of the listed directories, then the directory specified by the
    \grcmd{grOptionMetricPath} variable is searched.

  \item[\texttt{grOptionTermSize}] (integer, default=100) This
    variable effects the display of 1- and 2-index objects using the
    \grcmd{grdisplay} command. If the size of individual components of
    such objects are smaller than this value, then the object will
    appear in the form of a matrix when displayed using
    \grcmd{grdisplay}. If the components are larger than the value of
    this variable, they are listed individually as they are for
    tensors with three or more indices. By setting this variable to
    some very high value the matrix form of display can be assured,
    while setting this value to zero will prevent matrix-type display
    of tensors. The size of object is measured in words as returned by
    the MapleV \grcmd{length} command.

  \item[\texttt{grOptionTrace}] (boolean, default=\texttt{true}) If
    this variable is set to \texttt{true}, then the names of objects
    calculated during the execution of a \grcmd{grcalc} or
    \grcmd{grcalcalter} command will be displayed as they are
    computed. Note, this option may not work for some MapleV windowed
    interfaces for which the output for the \grcmd{printf} command
    is not flushed until the calling procedure is completed.

  \item[\texttt{grOptionWindows}] (boolean, default=\texttt{true}) If
    this variable is set \texttt{true} the GRTensorII `windows'
    display routines are used in the output of tensor components via
    \grcmd{grdisplay}. These routines provide support for superscript
    and subscript of indices in windowed versions of MapleV. However,
    in text-based versions, they display a number of extraneous
    characters than confuse the output. This option should be set to
    \texttt{false} if text-based MapleV is being used to run
    GRTensorII.

  \item[\texttt{grOptionVerbose}] (boolean, default=\texttt{false})
    When this variable is set \texttt{true} a notification message is
    printed out as each component is calculated by \grcmd{grcalc} or
    \grcmd{grcalcalter}.
\end{description}
%
%------------------------------------------------------------------------------
\vfill
%\bibliographystyle{unsrt}
%\bibliography{grtensor}
%------------------------------------------------------------------------------
\pagebreak
\section*{Commands described in this booklet:}
  \noindent
  \grcmdline{grlib ( \grarg{libName} )}
   \dotfill \pageref{spec:grlib}\\

  \noindent
  \grcmdline{grtensor()}
   \dotfill \pageref{spec:grtensor}\\

  \noindent
  \grcmdline{groptions()}
   \dotfill \pageref{spec:groptions}\\
%
%------------------------------------------------------------------------------
\vfill
\large
\noindent The information contained in this booklet is also available from the
following online help pages:\\

\noindent\texttt{?grtensor}, \texttt{?groptions}, \texttt{?grlib},
\texttt{?libname}, \texttt{?grt\_commands}, \texttt{?grt\_objects},
\texttt{?grt\_basis}, \texttt{?grt\_operators}, \texttt{?grt\_invars}.
%
%------------------------------------------------------------------------------
\end{document}
%==============================================================================
