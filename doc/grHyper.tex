%=================================================================-*-LaTeX-*-==
% GRTensorIII 1. Manual
% Booklet G: Hypersurfaces
%
% Peter Musgrave
% Jan 2017
%==============================================================================
\documentclass{article}
\usepackage{amsmath}
% Maple package causes a failure of the title text to rotate
% can just return through the missing maplelatex stmts while building
% and document appears to be ok
%\usepackage{maplestd2e}
\usepackage{grtensor}
\usepackage{grbooks}
\usepackage{hyperref}
%------------------------------------------------------------------------------
\begin{document}
%\setlength{\footrulewidth}{\headrulewidth}
\grtitle{\grHyperTitle}
\grlabel{\grHyperLabel}
\grdate{Jan 2017}
\grtitlepage
%
%==============================================================================
% Body.
%==============================================================================
\copyrightpage
\noindent 
GRTensor III supports the definition and evaluation of hyper-surfaces within a spacetime, and the joining
of two spacetimes along a hyper-surface. It facilitates the evaluation of the Darmois-Israel jump conditions
and the determination of the hyper-surface evolution and properties of the shell (if one exists). 
Hyper-surfaces and junctions for timelike, spacelike and null shells are supported. This functionality was
initially provided in the GRJunction package (refs). The core objects in this package have been migrated 
into GRTensor III Version 1.1. The commands from that package have been re-designed to allow for
direct (non-interactive) definitions that allow worksheet recalculation in a natural way.

There are numerous text book treatments and review papers that describe the junction formalism in detail. This
booklet does not attempt to cover this material. We present a personal choice of references to establish notation
and object definitions.

A copy of the GRTensorIII source code is available online at 
\href{url}{http:/gitlab.com/grtensor/GRTensorIII}. Merge requests
will be considered. 

%
%------------------------------------------------------------------------------
\section{Non-Null Hypersurfaces}
%------------------------------------------------------------------------------


%------------------------------------------------------------------------------
\section{Junction Conditions}
%------------------------------------------------------------------------------

%------------------------------------------------------------------------------
\section{Null Hypersurfaces}
%------------------------------------------------------------------------------

%------------------------------------------------------------------------------
\vfill
\bibliographystyle{unsrt}
\bibliography{grtensor}
\end{document}
%==============================================================================
