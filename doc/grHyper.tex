%=================================================================-*-LaTeX-*-==
% GRTensorIII 1. Manual
% Booklet G: Hypersurfaces
%
% Peter Musgrave
% Jan 2017
%==============================================================================
\documentclass{article}
\usepackage{amsmath}
% Maple package causes a failure of the title text to rotate
% can just return through the missing maplelatex stmts while building
% and document appears to be ok
%\usepackage{maplestd2e}
\usepackage{grtensor}
\usepackage{grbooks}
\usepackage{hyperref}
%------------------------------------------------------------------------------
\begin{document}
%\setlength{\footrulewidth}{\headrulewidth}
\grtitle{\grHyperTitle}
\grlabel{\grHyperLabel}
\grdate{Jan 2017}
\grtitlepage
%
%==============================================================================
% Body.
%==============================================================================
\copyrightpage
\noindent 
GRTensor III supports the definition and evaluation of hyper-surfaces within a spacetime, and the junction
of two spacetimes along hyper-surfaces. It facilitates the evaluation of the Darmois-Israel matching conditions
and the determination of the hyper-surface evolution and properties of the shell (if one exists). 
Hyper-surfaces and junctions for timelike, spacelike and null shells are supported. This functionality was
initially provided in the GRJunction package. It is now part of GRTensorIII. The commands from that package have been re-designed to allow for
direct (non-interactive) definitions that allow worksheet recalculation in a natural way. \\

There are numerous text book treatments and review papers that describe the junction formalism in detail. This
booklet does not attempt to cover this material. We present a personal choice of references to establish notation
and object definitions. In most things we are guided by "A Relativist's Toolkit" by Eric Poisson \cite{poisson:2004}.
Some of the issues of developing computer algorithms for hypersurface and junctions are described in the original 
GRJunction papers \cite{musgravelake:1994, musgravelake:1997}. \\

A copy of the GRTensorIII source code is available online at 
\href{url}{http:/gitlab.com/grtensor/GRTensorIII}. Merge requests
will be considered. \\

%
%------------------------------------------------------------------------------
\section{Hypersurfaces}
%------------------------------------------------------------------------------
A hyper-surface $\Sigma$ in a spacetime $M$ is a 3-dimensional sub-space of $M$. The co-ordinates of $\Sigma$ are in 
general distinct from those in $M$. In practice it is common to use some of the same co-ordinates in both $M$ and $\Sigma$. 
Here we follow the definitions and conventions in \cite{toolkit} and label the co-ordinates on $\Sigma$ as $y^a$, using Roman indices on tensorial objects. The co-ordinates on $M$ are $x^\alpha$ and Greek indices are used. \\

A hypersurface can be defined by either a set of relations of the form $x^\alpha = f(y^a)$ or by a scalar function $\Phi(x^\alpha)$ that is zero on 
$\Sigma$. \\

%------------------------------------------------------------------------------
\subsection{Timelike and Spacelike Surfaces}

The basic vectors of the surface in $M$ are defined by:
\begin{center}
$e^{\alpha}_{a} = \frac{\partial x^{\alpha}}{\partial y^a}$
\end{center}

%These correspond to the GRTIII object \text\tt{es(bdn,dn)}. This object is defined in $M$. 

The intrinsic metric $g_{a b}$ on $\Sigma$ is defined by:
\begin{center}
$g_{a b} = g_{\alpha \beta} e^{\alpha}_a e^{\beta}_b$
\end{center}
Here we deviate slightly from the nomenclature of \cite{poisson:2004} retaining the object \text\tt{g(dn,dn)} for the intrinsic metric instead of $h_{a b}$. \\

The normal to the hypersurface can be specified explicitly or derived as the gradient of a scalar definition $\Phi(x^\alpha) = 0$ of the surface. GRTIII 
allows either approach. If a scalar surface definition is provided the normal is defined by:
\begin{center}
$n_\alpha = \frac{\epsilon \Phi_{,\alpha}}{\left| g^{\mu \nu} \Phi_{,\mu} \Phi_{,\nu} \right|}$
\end{center}
where $\epsilon=1$ for a timelike surface and $-1$ for a spacelike surface. \\

The extrinsic curvature \text\tt{K(dn,dn)} of the surface is determined by:
\begin{center}
$K_{a b} = \nabla_\alpha n_\beta e^\alpha_a e^\beta_b = n_\alpha \left( \frac{\partial^2 x^\alpha}{\partial y^a \partial y^b}
+ \Gamma^\alpha_{\mu \nu} \frac{\partial x^{\mu}}{\partial y^a} \frac{\partial x^{\nu}}{\partial y^b} \right)$
\end{center}

The contracted forms of the Gauss-Codazzi equations are provided. These are:
\begin{center}
$-2 \epsilon G_{\mu \nu} n^\mu n^\nu = R + \epsilon \left( K^{i j}K_{ij} -K^2 \right)$ \\
\end{center}
and 
\begin{center}
$G_{\mu \nu} e^\mu_a n^\nu = K^b_{a|b} - K_{,a}$
\end{center}
These equations using $G{\mu \nu}$ are identities and serve as useful validation check for the package implementation. Using them to 
examine the physics of a spacetime is done by using Einstein's equation and providing a specific form for $T{\mu \nu}$
by using \text\tt{grdef} to define \text\tt{T(dn,dn)} and provide a phenomenology. See the \text\tt{hyper\_frw\_constraint} worksheet
for an example (following section 3.6.2 in \cite{poisson:2004}). 

\begin{center}
$16 \pi T_{\mu \nu} n^\mu n^\nu = R + \epsilon \left( K^{i j}K_{ij} -K^2 \right)$ \\
\end{center}
and 
\begin{center}
$8 \pi T_{\mu \nu} e^\mu_a n^\nu = K^b_{a|b} - K_{,a}$
\end{center}

Note that these equations mix the contraction of objects in $M$ defined in the co-ordinates of $M$ into an equation 
defined on $\Sigma$. By default the 
calculation will not apply the equations restricting the objects in $M$ to $\Sigma$ but this is easily done by using the
\text\tt{gralter} command with the \text\tt{cons} argument. \\

The GRTensorIII objects relating to timelike/spacelike hypersurfaces are listed in Table~\ref{tab:ts1} and~\ref{tab:ts2}.
%\begin{table}[!htbp]
\begin{table}
  \begin{center}
    \begin{tabular}{ll}\hline\hline
      \textbf{GRTensorIII name} & \textbf{Common representation}\\ \hline
      \texttt{g(dn,dn)}        & $g_{ab} $  \\
      \texttt{K(dn,dn)}        & $K_{ab} $  \\      
      \texttt{Ksq}        & $K_{ab} K^{ab}$  \\
      \texttt{trK}        & $K_{a}^{a} $  \\
%      \texttt{GCeqn1}        & $R_{\alpha \beta \mu \nu} e^\alpha_a e^\beta_b e^\mu_c e^\nu_d =
%      			R_{a b c d} + \epsilon \left( K_{ab} K_{bc} - K_{ac} K_{bd} \right) $  \\
%     \texttt{GCeqn1RHS}        &  R_{a b c d} + \epsilon \left( K_{ab} K_{bc} - K_{ac} K_{bd} \right) $  \\
      \texttt{C1GeqnRHS}        & $R + \epsilon \left( K^{i j}K_{ij} -K^2 \right)$  \\
       \texttt{C1Geqn}        & $-2 \epsilon G_{\mu \nu} n^\mu n^\nu = R + \epsilon \left( K^{i j}K_{ij} -K^2 \right)$  \\
      \texttt{C1Teqn}        & $16 \pi T_{\mu \nu} n^\mu n^\nu = R + \epsilon \left( K^{i j}K_{ij} -K^2 \right)$  \\
      \texttt{C2GeqnRHS(dn)}        & $ K^b_{a|b} - K_{,a}$ \\
     \texttt{C2Geqn(dn)}        & $G_{\mu \nu} e^\mu_a n^\nu = K^b_{a|b} - K_{,a}$ \\
     \texttt{C2Teqn(dn)}        & $8 \pi T_{\mu \nu} e^\mu_a n^\nu = K^b_{a|b} - K_{,a}$ \\
    \end{tabular}
    \caption{GRTensorIII objects defined on $\Sigma$ for timelike/spacelike surfaces}
    \label{tab:ts1}
  \end{center}
\end{table}

Objects defined on $M$:
\begin{table}
  \begin{center}
    \begin{tabular}{ll}\hline\hline
      \textbf{GRTensorIII name} & \textbf{Common representation}\\ \hline
%      \texttt{e(bdn,up)}        & $e^\alpha_a $  \\
      \texttt{Gnn}        & $G^{\mu \nu} n_\mu n_\nu $  \\
      \texttt{Gxn(dn)}        & $G_{\mu \nu} \frac{\partial x^\mu}{\partial y^a} n^\nu $  \\
      \texttt{Tnn}        & $T^{\mu \nu} n_\mu n_\nu $  \\
      \texttt{Txn(dn)}        & $T_{\mu \nu} \frac{\partial x^\mu}{\partial y^a} n^\nu $  \\
      \texttt{n(dn) }      & $n_\alpha$, normal to the surface   \\    
      \texttt{xform(up)}        & $x^\alpha(y^a)$, definition of the surface  \\    
    \end{tabular}
    \caption{GRTensorIII objects defined on $M$ for timelike/spacelike surfaces}
    \label{tab:ts2}
  \end{center}
\end{table}

%------------------------------------------------------------------------------
\subsection{Restricting components to $\Sigma$}
The formalization of hypersurface definition in a computer algebra system such as Maple highlights the
fact that it is common in the literature to be inconsistent in the application of the surface definition where
doing so complicates the expressions or makes them less intuitive. 

For example, a null surface metric may be:
\begin{center}
$\sigma_{AB} d\theta^A d\theta^B = \lambda^2 \left(d\theta^2 + sin^2 \theta d\phi^2 \right)$
\end{center}
but the extrinsic curvature is written as 
\begin{center}
$C_{AB} = \frac{1}{2r} \sigma_{AB}$
\end{center}
where the surface definition includes $r=-\lambda$.

In some cases the direct substitution of the $x^\alpha(y^a)$ is desired, but not in all cases. In addition there are cases where
a direct substitution will cause issues with Maple. Consider a term:
\begin{center}
\text\tt{diff(f(r), r)}
\end{center}
in the case where $r=-lambda$. Evaluating this in Maple:
\begin{center}
\begin{verbatim}
eval(subs(r = -lambda, diff(f(r), r)));
Error, invalid input: diff received -lambda, which is not valid for its 2nd argument
\end{verbatim}
\end{center}
Maple expects the argument to a \text\tt{diff} function to be a name, not an expression. \\

GRTensorIII will restrict components in $M$ to $\Sigma$ as follows:
\begin{enumerate}
\item Determine a list of direct substitutions: the $x^\alpha(y^a)$ where the RHS is one of the $y^a$
\item Map all derivatives that involve a non-direct substitution to the Maple \text\tt{Diff} function (the
inactive form of the \text\tt{diff} function). 
\item Freeze all the inactive \text\tt{Diff} functions so that a substitution of the $x^\alpha(y^a)$ will not 
be applied
\item Substitute the $x^\alpha(y^a)$
\item Un-freeze the \text\tt{Diff} functions
\end{enumerate}



%------------------------------------------------------------------------------
\subsection{Null Surfaces}

%------------------------------------------------------------------------------
\subsection{The \text\tt{hypersurf} command}

A hypersurface is defined in GRTensorIII with the \text\tt{hypersurf} command. 

%------------------------------------------------------------------------------
\section{Junction Conditions}
%------------------------------------------------------------------------------


%------------------------------------------------------------------------------
\vfill
\bibliographystyle{unsrt}
\bibliography{grtensor}
\end{document}
%==============================================================================
